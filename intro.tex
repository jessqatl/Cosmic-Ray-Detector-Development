\section{Introduction}

\label{intro}
Galactic cosmic rays are the high-energy particles that stream that into our solar system from the far bounds of our Galaxy as well as some low energy particles, associated with solar flares, from the Sun. 
%EH come back..
%The Earth atmosphere serves as an ideal detector for the high energy cosmic rays (CRs) which interact with the air molecule nuclei causing propagation of extensive air showers \cite{bidgoli:observing}. 
The primary cosmic ray (CR) particles are mainly energetic protons ($>$90$\%$) and about 9 $\%$ alpha particles (remaining 1$\%$ are heavier nuclei), which originate from astrophysical events, such as galactic nuclei, gamma ray burst, and pulsars, accelerated by expanding clouds of gas and magnetic fields from supernova explosions \cite{Supernova_cosmic}, \cite{pdg:cosmic,dorman:crv}. The primary CR particles interact with the molecules in the atmosphere and produce showers of secondary particles (mainly pions) at about 15 km altitude. These pions are decaying into muons which are the dominant cosmic ray particle radiation (about 80$\%$) at the surface of the Earth.  

Many interesting applications of the cosmic ray measurements have been discovered in recent years including the cosmic ray muon tomography for homeland security, volcanic activity monitoring and nuclear reactor core monitoring, etc. \cite{Pyramid_muon,Muon_tomography}. 
Probably the most significant application uses cosmic ray flux during seasonal changes to study the atmospheric and space weather. \cite{np6:phys} Numerous studies over the past decades report correlations between the dynamical changes of the earth's weather patterns and CR flux variation measured at the surface of the Earth \cite{kirkby:climate,lu:correlation,ollila:changes,shaviv:climate}. It has not yet been fully determined exactly how cosmic rays impact the earths climate; however, studies strongly suggest that the temperature of the earth follows more closely decade variations in galactic cosmic ray flux and solar cycle length than other solar activity parameters. \cite{svensmark:influence} 
The Nuclear Physics Group at Georgia State University (GSU) \cite{np6:phys} is currently developing novel, low-cost and portable cosmic ray detectors to be distributed around the world. One of the main goals of this project is to measure the cosmic ray radiation at the surface of the earth simultaneously at a global scale for studying the dynamical changes of the upper troposphere and the lower stratosphere. The success of this global measurement could lead to an unprecedented and accurate weather forecasting system both in short and long-term. 
%There are two computing related challenges for this project.  One is the need to monitor and collect data from the cosmic ray detector nodes in a world-wide cosmic ray detector network. The other is to systematically simulate the cosmic ray shower development in the atmosphere with variable geomagnetic field and atmospheric air density. 

In the presentation, I will present a new muon detector that I am building together with the graduate students in the Nuclear Physics Group. This detector is based on a design that was developed by the MIT group \cite{MIT_detector}, which consists of a 14 x 11 x 1 cm plastic scintillator paired with a 6 x 6 mm\(^2\) silicon photomultiplier (SiPM) to detect scintillation photons produced from muons that pass through the scintillator. My work mainly includes purchasing detector components, soldering circuit boards, assembling the detector and programming the readout electronics. I will also present the detector efficiency study in comparison with the existing muon detectors that have been built by the Nuclear Physics Group at GSU.

%An access to XSEDE's High Performance Computing (HPC) resources \cite{towns:xsede} has enabled us to run 
%extensive simulations that lead to more accurate results and comparison with the measured cosmic ray flux data
%at global scale.  

In the following sections, I present an overview of the most recent cosmic ray muon detector developed by the Nuclear Physics Group and the desktop muon detector that I have built based on the MIT design. The modifications and methods that were employed to the MIT design to allow the detector to be more applicable for cosmic ray research are presented as well. 