\section{Summary}
The MIT cosmic ray muon detector provided measurements efficiently and simply. The simple design makes for a quick build at a low price; we were able to build our detector for under \$70 each. However, the Arduino Nano does impose some restrictions. There is an uncertainty of +/- 50 ppm with the time of measurements, such that the first three muon counts are not accurate. The current code does not provide room for expansion of functionalities, as it currently occupies 60\% of memory space and 49\% of the dynamic memory while requiring that storage space and remaining dynamic memory be less that 70\%.  Components, such as the OLED screen and LED lights, are contributing to the significantly high dead time. To maintain their use while minimizing the dead time, we will need to add additional hardware to expand the capabilities of the detector. Connecting the Arduino Nano to another microcontroller, the Raspberry Pi, and modifying its code will eliminate these issues. Also, the Raspberry Pi has many built-in functionalities that could further enhance the detector to provide more meaningful measurements. The Raspberry Pi can work as the power supply for the detector, which will allow us to record data without having to monitor the readout on our laptop. Instead, data can be stored to an SD card on the Raspberry Pi and viewed later. These modifications will allow us to record muon occurrence for more than seven days. Through further analysis using an oscilloscope, I determined that the MIT design of directly coupling the SiPM to the scintillator provided data inconsistencies when oriented vertically. This finding gave myself, and my advisor Dr. He, the certainty that the GSU Muon Telescope is the best-suited detector for correlation study of cosmic ray flux and atmospheric and space weather. 